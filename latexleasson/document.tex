\documentclass{ctexart}
	% The document class with options:ctexbook,ctexreport,letter
\usepackage[T1]{fontenc}
%\usepackage[fontset=mac]{ctex}
\usepackage{graphicx}
\usepackage{amsmath}
\usepackage{amssymb}
\usepackage{natbib}
\newcommand{\eat}{我喜欢吃\heiti 肉}
\newcommand{\play}[2]{#1 喜欢玩 #2 \par}
\bibliographystyle{plainnat}%plain unsrt alpha abbrv
% A comment in the preamble
\author{meng~Xianglong\footnote{mxl} \and D.~Nobacon\footnote{MIT} \and Nan.Y\footnote{TJU}}
\title{\kaishu 杂谈勾股定理,\eat}
\date{\today}
%设置全局格式,内容与格式分离
\ctexset {		
	section = {
		name = {第,章},
		number = \chinese{section},
},
subsection = {
name = {,、}
%number = \arabic{subsection},
}
 }
%\newcommand\degree{^\circ}
\begin{document}
\maketitle
\tableofcontents

	\section{引言}
	\section{实验方案}l
	\subsection{实验方案一}
	%\subsubsection{1.1}
	\play{mxl}{gta}
	勾股定理可以用现代语言表述如下:
	
	直角三角形斜边的平方等于两腰的平方和
	
	可以用符号语言表述为:设直角三角形$ABC$,其中 $\angle C=90^\circ$,则有:
	\begin{equation}
		AB^2=BC^2+AC^2
	\end{equation}
	
	%字体族设置(罗马字体、无衬线字体、打字机字体)
	\textrm{Roman Famliy} \textsf{Sans Serif Family} \texttt{Typewriter Famliy}
	
	\rmfamily Roman Famliy {\sffamily Sans Serif Family} {\ttfamily Typewriter Famliy}
	
	%字体系列设置(粗细,宽度)
	\textmd{Medium Series}  \textbf{Boldface Series}%中等,加粗
	
	{\mdseries Medium Series} {\bfseries Boldface Series}
	
	%形状设置 直立文本,意大利斜体,伪斜体,小型大写
	{\upshape Upright Shape} {\itshape Italic Shape}  {\slshape Slanted Shape} {\scshape Small Caps Shape}
	
	中文字体的字体型号 {\heiti 黑体}$f^2(x)$
	
	\zihao{5} 你好
%中文中空格被忽略
%禁止使用中文全角空格,用\quad,\ ,代替
%特殊字符	
	\section{实验结果}
	As we konw television has been very popular with us.
	It is true that TV has some advantages. It makes our daily life more colourful.
	We can see movie at any time instead of going out.
	Just like a coin,everything has two sides.It also has some disadvantages.
	It makes us waste our time. 
	
	a \hfill b
	
具体的图片交叉引用使用了图\ref{brain}



	
     \LaTeX{}
\section{实验结论}
	` ' `` '' ``你好''
	
	- -- ---	

\subsection{结论一}
当然,在\LaTeX 中也可以使用如表\ref{exampletable}类型的表格
%表格
\begin{table}[htbp]%设置浮动体
	\centering
	\caption{示例}\label{exampletable}
	\begin{tabular}{|c|c|c|c|p{1.5cm}|}%{}列格式说明,p指定列宽度
		\hline
		姓名 & 语文 & 数学 & 外语 & 备注 \\
		\hline
		张三 & 87 & 100 & 93 & 优秀 \\
		\hline
	\end{tabular}
\end{table}
\section{希腊字母}
$\alpha$ $\theta$ $\uparrow$ $\beta$

$\sqrt[3]{x^2_{z}-y}$

$\frac{\sqrt[3]{x^2_{z}-y}}{\sqrt[3]{x^2_{m}-y}}$
\subsection{带编号可以交叉引用的公式}
\begin{equation}\label{1}
	\frac{\sqrt[3]{x^2_{z}-y}}{\sqrt[3]{x^2_{m}-y}}
\end{equation}

\subsection{不带编号可以交叉引用的公式}
\begin{equation*}\label{2}
	\sqrt[3]{x^2_{z}-y}
\end{equation*}

如公式\ref{1}是带编号的,公式\ref{2}是不带编号的

\section{矩阵}
%省略号可以用 \dots,\vdots,\ddots实现
$\begin{array}{cc}
	0 & 1 \\
	2 & 3
\end{array}$

$ \begin{matrix}
0 & 1 \\
2 & 3
\end{matrix}
 $\qquad
$$ \begin{pmatrix}
0 & \vdots \\
2 & 3
\end{pmatrix}
 $$
 $$ \begin{array}{*{20}{c}}
 1&1& \cdots &1\\
 2&3&3&2\\
 \vdots & \ddots &1& \vdots \\
 1&2& \cdots &3
 \end{array} $$
 $$ \left[ {\begin{array}{*{20}{c}}
 	1&1& \cdots &1\\
 	2&3&3&2\\
 	\vdots & \ddots &1& \vdots \\
 	1&2& \cdots &3
 	\end{array}} \right] $$
 $$ \left| {\begin{array}{*{20}{c}}
 	1&1& \cdots &1\\
 	2&3&3&2\\
 	\vdots & \ddots &1& \vdots \\
 	1&2& \cdots &3
 	\end{array}} \right| $$
 $$ \left( {\begin{array}{*{20}{c}}
 	{{a_{11}}}&{{a_{12}}}\\
 	{{a_{21}}}&{{a_{22}}}
 	\end{array}} \right)_{n \times n} $$
$$\begin{bmatrix}
	1 & 2 \\
	3 & 4
\end{bmatrix}$$

$$\begin{Bmatrix}
1 & 2 \\
3 & 4
\end{Bmatrix}$$

 $$\begin{vmatrix}
 	12 & 2 \\
 	3 & 4
 \end{vmatrix}$$

%三角矩阵
\[\begin{pmatrix}
a_{11} & a_{12}&\cdots&a_{1n} \\
&a_{22}& \cdots &a_{2n} \\
&     & \ddots & \vdots\\
\multicolumn{2}{c}{\raisebox{1.3ex}[0pt]{\huge 0}}
&    &a_{nn}
\end{pmatrix}\]

%行内小矩阵
复数 $z = f(x,y)$也可以用矩阵
\begin{math}
\left(
\begin{smallmatrix}
x & -y \\ y & x
\end{smallmatrix}
\right)	
\end{math}
来表示

\section{公式分行}
%gather和gather*命令
\begin{gather}
	a+b = b+a\\
	ab ba\\
	3 \times 5
\end{gather}

%align 和 align*环境(用&进行对齐)
\begin{align}
	x &= t+ \cos t +1 \\
	y &= 2\sin t
\end{align}
\begin{align*}
x &= t& x &= \cos t &x &= t \\
y &= 2t & y &=\sin(t+1) &y &=\sin x
\end{align*}

\[x = \left\{ \begin{array}{l}
5,\quad if \quad x = y\\
6,\quad if\quad x > y
\end{array} \right.\]

\begin{align}
x &= \left\{ \begin{array}{l}
5,\quad if \quad x = y\\
6,\quad if\quad x > y
\end{array} \right. \\
D(x) &= \begin{cases}
1,& \text{如果 }x \in \mathbb{Q};\\
0,& \text{如果 }x \in \mathbb{R}\setminus\mathbb{Q}
\end{cases}
\end{align}

%cases环境
\begin{equation}
	D(x) = \begin{cases}
	1,& \text{如果 }x \in \mathbb{Q};\\
	0,& \text{如果 }x \in \mathbb{R}\setminus\mathbb{Q}
	\end{cases}
\end{equation}
\section{参考文献使用}
这是一个参考文献引用\cite{dosovitskiy2020an}, 这是另外一个引用\cite{junnan2020prototypical},这是第三个引用\cite{liu_representation_2020}
这是来自导出的条目的引用\cite{choromanski2020rethinking}
\cite{chollet2015keras}
\nocite{*}%显示未引用参考文献
\bibliography{test,test1}
\end{document}